\documentclass{cleancv}

% Information information to create header:

\cvpicture{inti.jpg}
\firstname{Inti Manuel}
\lastname{Yabar-Pagaza}
\jobtitle{Computational biologist}
\phone{+45 27 47 29 54}
\email{intipagaza@live.dk}
\linkedin{linkedin.com/in/imyp} 
\github{github.com/imyp}
\webpage{imyp.github.io}
\location{Lyngby, Denmark} 

\begin{document}

% Create header
\makeheader


% Create right section
\begin{rightsection}

  % IT Skills
  \begin{itskills}
    \hardskill{Python}{5}%
    {%
      Experience with web apps (Flask, Dash), data analysis and
      visualization (pandas, matplotlib), and creating machine
      learning models (Pytorch).%
    }

    \hardskill{SQL}{4}%
    {%
      Experience using MySQL databases connected to web apps.%
    }

    \hardskill{Bash}{3}%
    {I have been using linux systems and the bash shell language daily
      for 5+ years.%
    }
    
    \hardskill{JSL}{3}%
    {
      I have created several scripts to automate the analysis of high
      throughput screening data.%
    }
    
    \hardskill{VBA}{3}%
    {
      I have used VBA to automate data formatting for input to
      liquid handling robots working in Novozymes.%
     }
    
    \hardskill{R}{3}%
    {
      During the course \emph{Statistics for Biochemistry} I analyzed
      data using various statistical models in R.%
    }
    
    \hardskill{HTML and CSS}{2}%
    {%
      I have experience creating simple pages and
      stylesheets for project pages.%
    }
    
  \end{itskills}
  
  \begin{certificates}
    \softskill{IBM AI Engineering Specialization}%
    {%
      at \textbf{Coursera}\vspace{0.5ex}\newline
      I completed a 6 course Professional Certificate and gained a
      practical understanding of \makebox{\textbf{Machine Learning}}
      (ML) \& \makebox{\textbf{Deep Learning}} (DL). This provided me
      with the technical skills to build AI systems,
      and :
      \begin{itemize}
      \item Implement ML algorithms including Classification,
        Regression, Clustering, and Dimensional Reduction using scipy
        \& scikitlearn.
      \item Perform ML on Big Data and deploy ML Algorithms and
        Pipelines on Apache Spark.
      \item Demonstrate understanding of Deep Learning models such as
        autoencoders, restricted Boltzmann machines,
        convolutional networks, recursive neural networks, and
        recurrent networks.
    \item Build deep learning models and neural networks using
      Keras,PyTorch and Tensorflow libraries.
    \item Demonstrate ability to present and communicate outcomes of
      deep learning projects.
    \end{itemize}

    Verify at: 
    \href{https://coursera.org/verify/professional-cert/HSNG3RS3VPJL}%
    {coursera.org/verify/professional-cert/HSNG3RS3VPJL}%
  }
  \end{certificates}
   
\end{rightsection}


% Profile
\profile{%
  I am a computational biologist with a background in biochemistry and
  a strong interest for math, physics and data science. I have
  experience with automation of data analysis, and enjoy teaching and
  communicating scientific and technical topics.\vspace{1ex}
  
  I find programming great for creating insightful and useful tools
  and I enjoy building programs that create value for many users.%
}

\begin{textblock}{1}(18,28.5)
  {\bodytext{Page 1/2}}
\end{textblock}


% WORK EXPERIENCE
\begin{expsec}

  % DTU
  % Department of Biotechnology and Biomedicine
  % Section of Protein Chemistry and Enzyme Technology
  % Protein Biophysics Group
  \cvitem{2019-2020}{Research assistant}%
  {Technical University of Denmark}{Lyngby, Denmark}%
  {%
    Computational and experimental investigation of biophysical
    properties of amyloid fibrils in the Protein Biophysics Group at
    Bioengineering led by Prof. Alexander K. Büll.

    \begin{itemize}
    \item calculated structural and energetic changes in amyloid
      fibrils upon mutations using Monte Carlo-based methods in
      Rosetta and PyRosetta.

    \item analyzed temperature and chemical dependence of fibril
      dissociation.

    \item set up protocols to automate liquid handling for high
      throughput biophysical experiments.

    \end{itemize}
    %
  }

  % Novozymes
  \cvitem[5.55cm]{2014-2018}{Student helper}%
  {Novozymes}{Bagsværd, Denmark}%
  {%
    \begin{itemize}
    \item developed web application to facilitate overview of
      progress of gene library samples using Python and SQL.
    \item automated data analysis and visualization for numerous
      projects involving assays using JMP Scripting Language.
    \item created parsers to unify data in standard complying format
      extracting metadata.
    \end{itemize}
    %
  }
  % KU
  \cvitem[4.7cm]{2014-2017}{Assistant teacher}%
  {University of Copenhagen}{Copenhagen, Denmark}%
  {%
    Assisted teaching in Biochemical Student Cafe and the summer
    Brush-up course for new attending biochemical students.

    \begin{itemize}
    \item taught and organized a week-long intensive math and
      chemistry course aimed to brush-up new students that are going
      to study biochemistry.
    \item helped first-year biochemistry students with weekly problem
      sets in mathematics, biology, organic chemistry and
      biochemistry.
    \item created explanatory solutions for problem sets for fellow
      assistant teachers.
    \end{itemize}
    %
  }

\end{expsec} 



% EDUCATION

\begin{edusec}

  % Master's degree
  \cvitem{2016-2019}{MSc in Biochemistry (Protein Chemistry)}%
  {University of Copenhagen}{Copenhagen, Denmark}%
  {%
    Courses in bioinformatics and biophysics with focus on
    structural biology.\vspace{1ex}

    My thesis, \textit{Characterization of San1p}, was supervised by
    Prof. K. Lindorff-Larsen, and focused on biophysical and bioinformatic
    characterization of a yeast ubiquitin ligase.\vspace{1ex}

    During the biophysical characterization I carried out nuclear
    magnetic resonance spectroscopy and nano differential scanning
    fluorimetry.\vspace{1ex}

    In the bioinformatic part of the I carried out coevolution
    analysis on the protein and analyzed the protein family of San1p
    using profile hidden Markov models.%
  }

% Bachelor's degree

  \cvitem[5.7cm]{2013-2016} {BSc In Biochemistry}%
  {University of Copenhagen}{Copenhagen, Denmark}%
  {%
    Courses in mathematics, physical chemistry, statistics and
    bioinformatics.\vspace{1ex}

    My thesis, \textit{Understanding and Mapping Mutations in HIV-1
      Protease}, was supervised by prof. K. Lindorff-Larsen, and
    focused on predicting the resistance of HIV-1 protease mutants to
    protease inhibitors used in HIV mutations.\vspace{1ex}

    During the project I worked with the Rosetta software suite to
    predict ΔΔG of mutated protease structures and compared with
    experimental data from a resistance assay and clinical databases.%
  }
\end{edusec}

\begin{volsec}
  \cvitem{2019-2020}{Math instructor}%
  {Matematik Center}{Copenhagen}%
  {%
    I weekly help middle and high school students with understanding
    mathematical concepts and solving mathematical problems.
    %
  }
  \cvitem[2.6cm]{2014-2015}{Introductory tutor}%
  {University of Copenhagen}{Copenhagen, Denmark}%
  {%
   organized social and academic events for 100+ biochemistry
   students throughout the academic year.%
  }
\end{volsec}

% PROJECTS
\begin{prosec}

  % HMMLogo and HMMVisualizer
  \projectitem{HMMLogo}%
  {%
    Visualize profile hidden Markov models (pHMMs) in Python using the
    \mbox{HMMLogo} package. \vspace{1ex}
    
    \href{https://hmmlogo.readthedocs.io}{hmmlogo.readthedocs.io} }%
  {HMMLogo visualizer}{%
    Flask app that showcases functionality of
    HMMLogo. Where you can
    visualize pHMMs. \vspace{1ex}\\
    \href{https://imyptest.herokuapp.com}{imyptest.herokuapp.com}
  }
\end{prosec}

\begin{rightsection}
  
  \begin{langsec}
    \softskill{Danish}
    {Native speaker. Born and raised in Denmark.}

    \softskill{English}{%
      Fluent proficiency written and spoken.
    }
    
    \softskill{Spanish}{%
      Native speaker with peruvian descent.
    }
  \end{langsec}
   
  \begin{personal}
    \softskill{Analytical}%
    {
      I enjoy problem solving, finding edge cases and studying complex systems.
    }
    \softskill{Curious}%
    {
      I enjoy learning about new topics and understand how processes work.%
    }
    \softskill{Cooperative}%
    {I enjoy collaborations, reciving and giving feedback to colleagues.}
    
  \end{personal}

  \begin{academic}
    \softskill{Project management}{%
      I can formulate, structure and carry out a project that aims out
      to question answer a scientific question.%
    }

    \softskill{Research}{%
      I can develop and apply biological and computational methods to
      generate new knowledge.%
    }

    \softskill{Data analysis}{%
      I can generate, assess, and analyze data and find potential
      sources of error, relevance of applied methods and validity of
      data.%
    }
  \end{academic}

  \begin{refsec}
    \softskill{Kresten Lindorff-Larsen}{%
      Professor of Biomolecular Sciences\newline
      Department of Biology\newline 
      University of Copenhagen%
    }

    \softskill{Alexander K. Büll}{%
      Professor of Protein Biophysics\newline
      Department of Biotechnology and Biomedicine\newline
      Technical University of Denmark%
      % 
    }
  \end{refsec}
\end{rightsection}
\makeheader
\begin{textblock}{1}(18,28.5)
  {\bodytext{Page 2/2}}
\end{textblock}
\end{document}

%%% Local Variables:
%%% TeX-engine: xetex
%%% mode: latex
%%% TeX-master: t
%%% End:
