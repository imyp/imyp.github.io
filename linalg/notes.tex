% Created 2019-07-09 Tue 19:29
% Intended LaTeX compiler: pdflatex
\documentclass[11pt]{article}
\usepackage[utf8]{inputenc}
\usepackage[T1]{fontenc}
\usepackage{graphicx}
\usepackage{grffile}
\usepackage{longtable}
\usepackage{wrapfig}
\usepackage{rotating}
\usepackage[normalem]{ulem}
\usepackage{amsmath}
\usepackage{textcomp}
\usepackage{amssymb}
\usepackage{capt-of}
\usepackage{hyperref}
\author{imyp}
\date{\today}
\title{Notes on linear algebra}
\hypersetup{
 pdfauthor={imyp},
 pdftitle={Notes on linear algebra},
 pdfkeywords={},
 pdfsubject={},
 pdfcreator={Emacs 25.2.2 (Org mode 9.2.3)}, 
 pdflang={English}}
\begin{document}

\maketitle

\section*{Definition of a linear space}
\label{sec:org8c92a3f}
\begin{quote}
The concept of a linear space generalizes that of the set of all vectors.
\end{quote}
A set \(\mathbf{K}\) is called a \emph{linear} (or \emph{affine}) \emph{space over a
field} \(K\) if 
\begin{enumerate}
\item Given two elements \(x,y \in \mathbf{K}\), there is a rule (the
addition rule) leading to a (unique) element \(x+y\in \mathbf{K}\),
called the \emph{sum} of \(x\) and \(y\).
\item Given any element \(x\in \mathbf{K}\) and any number \(\lambda \in K\),
there is a rule (the rule for multiplication by a number) leading to
a (unique) element \(\lambda x \in \mathbf{K}\), called the \emph{product}
of the element \(x\) and the number \(\lambda\).
\item These two rules obey 8 specific axioms.
\end{enumerate}

\section*{Properties of the addition rule}
\label{sec:orga39333f}

\begin{enumerate}
\item \(x+y=y+x\) for every \(x,y\in \mathbf{K}\) (commutative);
\item \((x+y)+z=x+(y+z)\) for every \(x,y,z \in \mathbf{K}\) (associative);
\item There exists an element \(0\in \mathbf{K}\) (the \emph{zero vector}) such
that \(x+0=x\) for every \(x\in\mathbf{K}\);
\item For every \(x\in\mathbf{K}\) there exists an element \(y\in\mathbf{K}\)
(the \emph{negative element}) such that \(x+y=0\).
\end{enumerate}

\section*{Properties of the addition rule}
\label{sec:orgfd9f181}

Commutative:
$$x+y=y+x\text{ for every }x,y\in \mathbf{K}$$

Associative:
$$(x+y)+z=x+(y+z)\text{ for every }x,y,z \in \mathbf{K}$$

Identity element:
$$\text{An element }\mathbf{0}\in\mathbf{K}\text{ exists, such
that }x+\mathbf{0}=x\text{ for every }x\in K$$

Inverse element:
$$\text{For every }x\in\mathbf{K}\text{ there exist an element
}y\in\mathbf{K}\text{ such that }x+y=\mathbf{0}$$
\section*{Properties of the rule for multiplication by a number}
\label{sec:org804fbc8}
\begin{enumerate}
\item \(1\cdot x=x\) for every \(x\in\mathbf{K}\);
\item \(\alpha(\beta x)=(\alpha \beta)x\) for every \(x\in\mathbf{K}\) and
every \(\alpha,\beta \in K\);
\item \((\alpha+\beta)x = \alpha x + \beta x\) for every \(x\in\mathbf{K}\)
and every \(\alpha,\beta \in K\);
\item \(\alpha(x+y)=\alpha x+\alpha y\) for every \(x,y \in \mathbf{K}\) and
every \(\alpha \in K\).
\end{enumerate}
\section*{Properties of multiplication}
\label{sec:org5907f68}

Identity element:
$$ 1 \cdot x = x\text{ for every } x\in\mathbf{K}$$
Associative: 
$$ \alpha(\beta x)=(\alpha \beta) x \text{ for every } x\in\mathbf{K}
\text{ and every } \alpha,\beta\in K$$
Distributive with respect to \(\mathbf{K}\):
$$(\alpha+\beta)x=\alpha x + \beta x \text{ for every } x\in\mathbf{K}
\text{ and every } \alpha,\beta\in K$$
Distributive with respect to \(K\):
$$\alpha(x+y)=\alpha x+\alpha y \text{ for every } x,y \in\mathbf{K}
\text{ and every } \alpha \in K$$

\section*{Linear dependence}
\label{sec:org7b98bc4}
Let \(x_1,x_2,\ldots, x_k\) be vectors of the linear space \(\mathbf{K}\)
over a field \(K\).
Let \(\alpha_1,\alpha_2,\ldots, \alpha_k\) be numbers from \(K\).
Then the vector \(\boxed{y=\alpha_1 x_1 + \alpha_2 x_2 +\cdots +
\alpha_k x_k}\) is called a \emph{linear combination} of the vectors \(x_1,
x_2, \ldots , x_k\).
The numbers \(\alpha_1,\alpha_2,\ldots,\alpha_k\) are called the
\emph{coefficients} of the linear combination.
\end{document}